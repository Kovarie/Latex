\documentclass  [12pt]{article}
\usepackage{graphicx}
\usepackage[magyar]{babel}
\usepackage[T1]{fontenc}

\author {WIKIPEDIA} 
\date{2021.10.16.}
\title{Minimax elv }

\begin{document}

	\begin{titlepage}
    		\maketitle
	\end{titlepage}



A minimax elv a döntéselméletben, a játékelméletben és a statisztikában alkalmazott döntési szabály, ami szerint azt a lehetőséget kell választani, ami minimalizálja a maximális veszteséget. Felfogható a minimális nyereség maximalizálásaként is.

Az elmélet megalkotása a kétfős zéróösszegű játékokkal kezdődött, ami magába foglalta a két játékos szimultán döntéseit és a felváltva tett lépéseit is, aztán bonyolultabb esetekre is kiterjesztették.

A játékelméletben
A minimax elv egy kevert stratégia, ami része a zéró összegű játékok megoldásának. A zéró összegű játékokban a minimax elv megadja a nyeregpontot.

Definíció: Legyen A és B nem üres halmaz,   adott függvény. Az   pont nyeregpont, ha minden  -ra és minden  -re  
Neumann-tétel: Minden olyan kétszemélyes, zéró összegű játéknak van nyeregpontja, amiben véges sok elemi stratégia van. Azaz a legjobb kevert stratégiával az első játékos várható nyereségének maximuma egyenlő a második játékos várható veszteségének minimumával.

Neumann János[1] szerint nem érdemes játékelméletet csinálni a minimax tételnek is nevezett tétel nélkül.[2]

A tétel általánosításai Sion minimaxtétele és Parthasarathy tétele.

Neumann-tétel
Definíció: Egy vektor tetején a vektor legnagyobb koordinátáját értjük; a vektor alja a legkisebb koordinátája.

Neumann-tétel: Minden olyan kétszemélyes, zéró összegű játéknak van nyeregpontja, amiben véges sok elemi stratégia van. Azaz a legjobb kevert stratégiával az első játékos várható nyereségének maximuma egyenlő a második játékos várható veszteségének minimumával.

A tétel egy másik alakban: Jelölje A a játék kifizetési mátrixát. Az A oszlopai által feszített politópban levő elemek minimuma egyenlő az A sorai által feszített politópban levő elemek aljának minimumával.

Bizonyítás: a dualitástétel segítségével.


%Példa
%\includegraphics[height=4cm]{fa.png}


\begin{figure}

	\centering
		\includegraphics[width=0.5\textwidth]{fa.png}
	\caption{Példa}
 
	\label{ref:Példa}

\end{figure}

\pagebreak


Maximin

Sokszor előfordul a játékelméletben, hogy a maximin különbözik a minimaxtól. A zéró összegű játékok elméletében a minimax az ellenfél nyereségének minimalizálását jelöli, ami a zéró összegű játékok esetén a saját veszteség minimalizálásának, azaz a saját nyereség maximalizálásának felel meg.
A maximin a nem zéró összegű játékok esetén használatos stratégiát jelenti, ami maximalizálja a saját nyereséget. Ez általában nem egyezik az ellenfél nyereségének a minimumával, vagy a nyeregponti stratégiával.


\begin{table}[b]

	\centering
	\begin{tabular}{ | l | c | c | r |}
   	\hline
       x  & B1 &  B2 & B3 \\ \hline
    A1  & +3  &  -2 &  +2 \\ \hline
    A2  & -1  &    0 &  +4 \\ \hline
    A3  & -4  &   -3 &  +1 \\

	\hline

	\end{tabular}
	\caption{}


\end{table} 





Példa

A következő példa egy zéró összegű játék. A két játékos, A és B szimultán lép.
Tegyük fel, hogy a játék kifizetési mátrixa az A játékos számára a fenti mátrix, és a B játékos számára az előjelek fordítottak. Ekkor A minimax választása A2, mivel itt a legrosszabb esetben 1-et kell fizetni, és B minimax választása B2, mert ekkor a legrosszabb esetben nincs nyeremény.
Ez a megoldás nem stabil. Ha B azt hiszi, hogy A A2-t választja, akkor B1 mellett dönt. Ha A azt hiszi, hogy B B1 mellett dönt, akkor megjátssza A1-et. Ha B azt hiszi, hogy A A1-et játssza meg, akkor B2-t választja. A determinisztikus stratégia kiismerhető, ezért csak nem determinisztikus stratégia lehet stabil.
A stabil kevert stratégiák: A 1/6 valószínűséggel választja A1-et, és 5/6 valószínűséggel A2-t. B 1/3 valószínűséggel választja B1-et, és 2/3 valószínűséggel B2-t. Ezek a stratégiák stabilak, és nem javíthatók.



\end{document}
